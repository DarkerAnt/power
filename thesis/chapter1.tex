\chapter{Motivation}
The annual cost of blackouts in the United States is estimated to be \$80 billion.\cite{lacommare2004understanding} Limiting the cost of blackouts would be of great benefit to society. Each blackout event has a specific set of conditions and causes that result in a cascading failure. A single blackout event provides little information about the nature of future blackouts. Thus it is important to view blackouts not as random events, but as part of a larger dynamical system.\cite{carreras2004complex} 

\chapter{Blackouts}
An electric power transmission system is designed to meet the demand for electricity with minimal disruption of service. Operators will go to great lengths to ensure high availability of the transmission system through design, management, and maintenance of the system. \cite{carreras2001evidence} However, the efforts to prevent small disturbances can increase the severity of large disturbances. [why?, cite]

When a component of the transmission system fails, the load on the component must be transferred to other parts of the network. As the system moves towards maximum operating capacity there exists fewer options to transfer load away from a local disturbance. When there are no available options to transfer the excess load then a cascading failure can occur. If a component fails and its neighboring components are unable to handle the additional load then the neighboring components will also fail. The load transferred from these failed components can cause additional neighboring components to fail. With each additional failure the network becomes less resilient to failure [cite]. In this manner a single failure can cascade into a system wide disruption of service. 


\chapter{Previous Work}
After compiling the North American Electrical Reliability Council (NERC) reports Carreras et al. found evidence suggesting electrical power networks are self organizing critical systems and exhibit a power tail.\cite{carreras2000initial, carreras2001evidence} The CASCADE \cite{dobson2002examining} model was devised to provide a high level probabilistic view of power network load transferring mechanics. The OPA \cite{dobson2001initial, carreras2004complex, newman2011exploring} \footnote{Oak Ridge National Laboratory, Power Systems Engineering Research Center at the University of Wisconsin, University of Alaska} model was created to examine the complex dynamics believed to be found in power networks. OPA uses two timescales to model changes in network capacity and cascading overloads. The Monte Carlo based Manchester model examines power networks at a more detailed level. The model includes a variety of AC dynamics not found in OPA such as generator instability and re-dispatch of active and reactive resources.\cite{nedic2006criticality, baldick2009vulnerability} 

\chapter{Current Methods}
The current blackout prevention method practiced by industry is N-x contingency analysis. N-1 contingency analysis ensures that no single initiating outage will result in a cascading blackout. The N-1 contingency does not asses the possibility of cascading failures caused by multiple, unrelated events. In practice the N-1 contingency is computed for each balancing authority individually and does not extend to the interconnects between BAs. The Northeast Blackout of 2003 involved multiple contingencies prior to cascading. This suggests a higher order than N-1 is required to prevent such occurrences.\cite{baldick2009vulnerability, liscouski2004final} The number of contingencies that must be computed for an N-x contingency analysis is exponential in x and has a large constant running time. An 150,000 element N-2 contingency analysis computed on 512 processors took 25.8 hours.\cite{huang2009massive} The current industry tool used to identify cascading failures is TRELSS\footnote{Transmission Reliabilty Evaluation of Large-Scale Systems} blah blah blah TRELSS.


\chapter{CASCADE Model}

\chapter{OPA Model}
The OPA model uses a slow and fast timescale to model power networks.\cite{carreras2004complex} The slow timescale models the growth of demand, increased generator capacity in response to demand, and increased line capacity in response to blackouts. The fast timescale uses linear programming to solve the dc power flow of a network and captures the fast dynamics such as cascading overloads and outages.\cite{carreras2004complex}
